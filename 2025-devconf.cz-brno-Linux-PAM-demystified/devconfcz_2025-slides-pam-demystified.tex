% Copyright (C) 2012-2025 Dmitry V. Levin <ldv@strace.io>
% Permission is granted to copy, distribute and/or modify this document
% under the terms of the GNU Free Documentation License, Version 1.2
% or any later version published by the Free Software Foundation;
% with no Invariant Sections, no Front-Cover Texts, and no Back-Cover Texts.

\documentclass[unicode,aspectratio=169,xcolor={table,dvipsnames,usernames}]{beamer}

\usepackage[utf8]{inputenc}
\usepackage[T2A]{fontenc}
\usepackage[english]{babel}
\usepackage{alltt}

\colorlet{darkred}{BrickRed}
\colorlet{darkgreen}{OliveGreen}
\newcommand{\symlinebreak}{\textcolor{blue}{\(\hookleftarrow\)}}
\newcommand{\symlinecont}{\textcolor{blue}{\(\longrightarrow\)}}

\pgfdeclareimage[height=0.6cm]{left-corner-logo}{devconf-cz-icon.pdf}
\pgfdeclareimage[height=0.6cm]{right-corner-logo}{pamela.jpg}
\pgfdeclareimage[height=2cm]{title-logo}{devconf-cz.pdf}
\pgfdeclareimage[height=4cm]{pam-logo}{pamela.jpg}
\pgfdeclareimage[height=4cm]{strace-logo}{strace-straus.pdf}

\usetheme{Warsaw}
\setbeamertemplate{headline}{}
\setbeamertemplate{footline}{\pgfuseimage{left-corner-logo} \hfill \pgfuseimage{right-corner-logo}}
\setbeamertemplate{navigation symbols}{}
\setbeamertemplate{logo}{}
\setbeamercovered{transparent}

\title{\Huge Linux-PAM demystified}
\author{\LARGE Dmitry~Levin}
\date{\Large Brno, 2025}
\titlegraphic{\pgfuseimage{title-logo}}

\begin{document}

%%%%%%%
\begin{frame}[noframenumbering]
\titlepage
\end{frame}

%%%%%%%
\begin{frame}{\Large Agenda\hfill [\insertframenumber/\inserttotalframenumber]}
\Large
\begin{itemize}
	\item What and Why
	\item Core concepts
	\item Configuration files and modules
	\item Configuration rules syntax
	\item Configuration rules control values
	\item Frozen stack
	\item Troubleshooting and Best Practices
\end{itemize}
\end{frame}

%%%%%%%
\begin{frame}{\Large What is Linux-PAM?\hfill [\insertframenumber/\inserttotalframenumber]}
\Large
\begin{block}{Framework}
\begin{itemize}
	\item shared libraries: libpam, libpamc, libpam\_misc
	\item Pluggable Authentication Modules: pam\_*.so
	\item configuration: /etc/pam.d/*
\end{itemize}
\end{block}
\begin{block}{Purpose}
\begin{itemize}
	\item Enable system administrators to choose how applications authenticate users.
	\item Switch between the authentication mechanisms without recompiling applications.
\end{itemize}
\end{block}
\end{frame}

%%%%%%%
\begin{frame}{\Large The "Before PAM" Problem\hfill [\insertframenumber/\inserttotalframenumber]}
\begin{block}{Hardcoded logic}
Every application that needed to authenticate a user had its own code \\
	to handle /etc/passwd, later also /etc/shadow.
\end{block}
\begin{block}{Rigidity}
To introduce a new authentication method like Kerberos or OTP, \\
	every application had to be modified and recompiled.
\end{block}
\begin{block}{Inconsistency}
Different applications might implement authentication, password checking, \\
	or account lockout rules slightly differently.
\end{block}
\begin{block}{Limited control}
Enforcing system-wide security policies, e.g. all interactive logins must use 2FA, \\
	is difficult.
\end{block}
\end{frame}

%%%%%%%
\begin{frame}{\Large PAM management groups\hfill [\insertframenumber/\inserttotalframenumber]}
\begin{block}{Authentication: Are you who you say you are?}
\begin{itemize}
	\item Verify the user's identity.
	\item Grant credentials.
\end{itemize}
\end{block}
\begin{block}{Account management: Are you {\bf allowed} to use this service {\bf right now}?}
\begin{itemize}
	\item Check account validity and restrictions.
\end{itemize}
\end{block}
\begin{block}{Session management: What needs to be set up for your session?}
\begin{itemize}
	\item Actions that need to occur before the service is granted.
	\item Actions that need to occur after the service termination.
\end{itemize}
\end{block}
\begin{block}{Password management: How can you change your authentication token?}
\begin{itemize}
	\item Prompting for a new password.
	\item Enforcing password quality rules.
	\item Changing the password.
\end{itemize}
\end{block}
\end{frame}

%%%%%%%
\begin{frame}{\Large PAM management groups: Authentication\hfill [\insertframenumber/\inserttotalframenumber]}
\Large
\begin{block}{Are you who you say you are?}
\begin{itemize}
	\item Checking a typed password against {\it /etc/shadow} \\ (e.g. via {\it pam\_unix.so}).
	\item Validating a U2F dongle (e.g. via {\it pam\_u2f.so}) \\ or a biometric scan (e.g. via {\it pam\_fprintd.so}).
	\item Querying a remote server for credentials \\ (e.g. via {\it pam\_sss.so}).
	\item Prompting for a One-Time Password.
\end{itemize}
\end{block}
\end{frame}

%%%%%%%
\begin{frame}{\Large PAM management groups: Account management\hfill [\insertframenumber/\inserttotalframenumber]}
\Large
\begin{block}{Are you {\bf allowed} to use this service {\bf right now}?}
\begin{itemize}
	\item Is the account enabled? Is the password not expired? \\ ({\it pam\_unix.so})
	\item Is the account locked due to too many failed login attempts? \\ ({\it pam\_faillock.so})
	\item Are there time-based restrictions on when this user can log in? \\ ({\it pam\_time.so})
	\item Is the user allowed for this service? \\ ({\it pam\_access.so})
\end{itemize}
\end{block}
\end{frame}

%%%%%%%
\begin{frame}{\Large PAM management groups: Session management\hfill [\insertframenumber/\inserttotalframenumber]}
\Large
\begin{block}{What needs to be set up for your session?}
\begin{itemize}
	\item Initialize kernel session keyring \\ ({\it pam\_keyinit.so}).
	\item Set resource limits \\ ({\it pam\_limits.so}).
	\item Set the file mode creation mask \\ ({\it pam\_umask.so}).
	\item Register user session in the login manager \\ ({\it pam\_systemd.so}).
	\item Create a home directory \\ ({\it pam\_mkhomedir.so}).
\end{itemize}
\end{block}
\end{frame}

%%%%%%%
\begin{frame}{\Large PAM management groups: Password management\hfill [\insertframenumber/\inserttotalframenumber]}
\Large
\begin{block}{How can you change your authentication token?}
\begin{itemize}
	\item Prompting for a new password.
	\item Enforcing password quality rules \\
		({\it pam\_passwdqc.so} or {\it pam\_pwquality.so}).
	\item Making sure the user does not use the same password \\
		too frequently ({\it pam\_pwhistory.so}).
	\item Changing the password \\
		({\it pam\_unix.so}).
\end{itemize}
\end{block}
\end{frame}

%%%%%%%
\begin{frame}{\Large Decoupling and Flexibility\hfill [\insertframenumber/\inserttotalframenumber]}
\begin{block}{Decoupling}
\begin{itemize}
	\item PAM separates the applications (like {\it login} or {\it sshd}) from the underlying authentication, account, session, and password management policies.
	\item Applications just talk to the PAM library.
	\item Administrators configure PAM.
\end{itemize}
\end{block}
\begin{block}{Applications talk to the PAM library}
\begin{tabular}{l|l|l}
	{\bf type} & {\bf API function name} & {\bf description} \\
\hline
	auth & pam\_authenticate & Authenticate this user \\
	auth & pam\_setcred & Manage credentials of this user \\
\hline
	account & pam\_acct\_mgmt & Check account validity and restrictions for this user \\
\hline
	password & pam\_chauthtok & Change the authentication token for this user \\
\hline
	session & pam\_open\_session & Set up a session for this user \\
	session & pam\_close\_session & End the session for this user \\
\end{tabular}
\end{block}
\end{frame}

%%%%%%%
\begin{frame}{\Large PAM configuration files\hfill [\insertframenumber/\inserttotalframenumber]}
\begin{block}{Service configuration files}
\begin{itemize}
	\item PAM configuration is service-specific.
	\item Configuration files are stored in {\bf /etc/pam.d/}.
	\item Service configuration files are named after services (like {\it login} or {\it sshd}).
	\item When a service, e.g. {\it login}, needs to authenticate a user, it tells the PAM library:
		I'm the login service, please handle this authentication based on my configuration.
	\item If a specific service file doesn't exist, or when it doesn't specify a management group,
		PAM falls back to a default configuration for this management group defined in /etc/pam.d/other,
		which usually denies access.
\end{itemize}
\end{block}
\begin{block}{Common configuration files}
\begin{itemize}
	\item conventionally stored in {\bf /etc/pam.d/}
	\item included by service-specific configuration files and other common configuration files
	\item used to implement system-wide policies
\end{itemize}
\end{block}
\end{frame}

%%%%%%%
\begin{frame}{\Large PAM Modules\hfill [\insertframenumber/\inserttotalframenumber]}
\Large
\begin{block}{Modules: the workhorses ({\it pam\_*.so})}
\begin{itemize}
	\item Modules are shared objects loaded dynamically \\ by the PAM library according to the service configuration.
	\item Typically located in /lib/security/ or /lib64/security/.
	\item Each module is designed to perform a specific task.
	\item There are many modules available:
	\item standard modules packaged along with the PAM library
	\item other modules provided by other packages
	\begin{description}
		\item[pam\_deny.so] always returns failure
		\item[pam\_permit.so] always returns access, useful as a placeholder
	\end{description}
\end{itemize}
\end{block}
\end{frame}

%%%%%%%
\begin{frame}[fragile]{\Large Configuration file syntax\hfill [\insertframenumber/\inserttotalframenumber]}
\begin{block}{PAM rule: {\bf type control module-path [module-arguments]}}
\begin{itemize}
	\item {\bf type}: the management group that the rule corresponds to
	\item {\bf control}: determines how the return value of this module affects the overall outcome for the management group
	\item {\bf module-path}: the filename of the PAM module to be used
	\item {\bf module-arguments}: optional arguments passed to the module
\end{itemize}
\end{block}
\scriptsize
\begin{block}{Example: /etc/pam.d/login (simplified)}
\begin{alltt}
auth       required     pam_unix.so nullok
account    required     pam_nologin.so
account    required     pam_unix.so
password   requisite    pam_passwdqc.so config=/etc/passwdqc.conf
password   required     pam_unix.so use_authtok shadow nullok
session    required     pam_loginuid.so
session    optional     pam_keyinit.so force revoke
session    required     pam_limits.so
-session   optional     pam_systemd.so
session    required     pam_unix.so
\end{alltt}
\end{block}
\end{frame}

%%%%%%%
\begin{frame}[fragile]{\Large PAM config rule control values: {\bf required} \hfill [\insertframenumber/\inserttotalframenumber]}
\begin{block}{{\bf required}}
\begin{itemize}
	\item Failure will lead to the PAM framework returning failure but only after the remaining stacked modules for this management group have been invoked.
\end{itemize}
\end{block}
\begin{block}{Example: /etc/pam.d/login (simplified)}
\begin{alltt}
auth       {\bf required}     pam_unix.so nullok
account    {\bf required}     pam_nologin.so
account    {\bf required}     pam_unix.so
password    requisite   pam_passwdqc.so config=/etc/passwdqc.conf
password    requisite   pam_pwhistory.so use_authtok
password   {\bf required}     pam_unix.so use_authtok shadow nullok
session    {\bf required}     pam_loginuid.so
session     optional    pam_keyinit.so force revoke
session    {\bf required}     pam_limits.so
-session    optional    pam_systemd.so
session    {\bf required}     pam_unix.so
\end{alltt}
\end{block}
\end{frame}

%%%%%%%
\begin{frame}[fragile]{\Large PAM config rule control values: {\bf requisite} \hfill [\insertframenumber/\inserttotalframenumber]}
\begin{block}{{\bf requisite}}
\begin{itemize}
	\item Like {\bf required}, however, in the case that this module returns a failure, \\
		control is directly returned to the application or to the superior PAM stack.
\end{itemize}
\end{block}
\begin{block}{Example: /etc/pam.d/login (simplified)}
\begin{alltt}
auth       required     pam_unix.so nullok
account    required     pam_nologin.so
account    required     pam_unix.so
password  {\bf requisite}      pam_passwdqc.so config=/etc/passwdqc.conf
password  {\bf requisite}      pam_pwhistory.so use_authtok
password   required     pam_unix.so use_authtok shadow nullok
session    required     pam_loginuid.so
session    optional     pam_keyinit.so force revoke
session    required     pam_limits.so
-session   optional     pam_systemd.so
session    required     pam_unix.so
\end{alltt}
\end{block}
\end{frame}

%%%%%%%
\begin{frame}[fragile]{\Large PAM config rule control values: {\bf sufficient} \hfill [\insertframenumber/\inserttotalframenumber]}
\begin{block}{{\bf sufficient}}
\begin{itemize}
	\item If the module succeeds and no prior {\bf required} module has failed, the PAM stack succeeds immediately without calling any further modules in the stack.
	\item Otherwise, the return value of the module is ignored and processing of the PAM module stack continues unaffected.
\end{itemize}
\end{block}
\begin{block}{Example: /etc/pam.d/su (simplified)}
\begin{alltt}
auth      {\bf sufficient}     pam_rootok.so
auth       required     pam_unix.so nullok
account   {\bf sufficient}     pam_succeed_if.so uid = 0 use_uid quiet
account    required     pam_unix.so
password   requisite    pam_passwdqc.so config=/etc/passwdqc.conf
password   required     pam_unix.so use_authtok shadow nullok
\ldots
\end{alltt}
\end{block}
\end{frame}

%%%%%%%
\begin{frame}[fragile]{\Large PAM config rule control values: {\bf optional} \hfill [\insertframenumber/\inserttotalframenumber]}
\begin{block}{{\bf optional}}
\begin{itemize}
	\item The success or failure of this module is only important if it is the only module in the stack associated with this management group.
\end{itemize}
\end{block}
\begin{block}{Example: /etc/pam.d/su (simplified)}
\begin{alltt}
auth       sufficient   pam_rootok.so
auth       required     pam_unix.so nullok
account    sufficient   pam_succeed_if.so uid = 0 use_uid quiet
account    required     pam_unix.so
password   requisite    pam_passwdqc.so config=/etc/passwdqc.conf
password   required     pam_unix.so use_authtok shadow nullok
session    {\bf optional}     pam_keyinit.so revoke
session    required     pam_limits.so
-session   {\bf optional}     pam_systemd.so
session    required     pam_unix.so
session    {\bf optional}     pam_xauth.so
\end{alltt}
\end{block}
\end{frame}

%%%%%%%
\begin{frame}[fragile]{\Large PAM config rule control values: {\bf include} \hfill [\insertframenumber/\inserttotalframenumber]}
\begin{block}{{\bf include}}
\begin{itemize}
	\item Include all lines of the same type from the configuration file \\
		specified as an argument to this control.
\end{itemize}
\end{block}
\begin{block}{Example: /etc/pam.d/su}
\begin{alltt}
auth            sufficient      pam_rootok.so
auth            required        pam_wheel.so use_uid
auth            substack        system-auth
auth           {\bf include}          postlogin
account         sufficient      pam_succeed_if.so uid = 0 use_uid quiet
account        {\bf include}          system-auth
password       {\bf include}          system-auth
session        {\bf include}          system-auth
session        {\bf include}          postlogin
session         optional        pam_xauth.so
\end{alltt}
\end{block}
\end{frame}

%%%%%%%
\begin{frame}[fragile]{\Large PAM config rule control values: {\bf substack} \hfill [\insertframenumber/\inserttotalframenumber]}
\begin{block}{{\bf substack}}
\begin{itemize}
	\item Include all lines of the same type from the configuration file \\
		specified as an argument to this control.
	\item {\bf requisite} and {\bf sufficient} in a substack does not cause skipping \\
		the rest of the complete module stack, but only of the substack.
	\item Jumps in a substack also can not jump out of it.
	\item The whole substack is counted as one module \\
		when the jump is done in a parent stack.
\end{itemize}
\end{block}
\begin{block}{Example: /etc/pam.d/su (excerpt)}
\begin{alltt}
auth            sufficient      pam_rootok.so
auth           {\bf substack}         system-auth
auth            include         postlogin
\ldots
\end{alltt}
\end{block}
\end{frame}

%%%%%%%
\begin{frame}{\Large PAM config rule control values: advanced syntax \hfill [\insertframenumber/\inserttotalframenumber]}
\begin{block}{The syntax: [{\it value1}={\it action1} {\it value2}={\it action2} \ldots {\it valueN}={\it actionN}]}
\begin{description}
	\item[valueN] corresponds to the return value returned by the module
	\item[actionN] specifies the action
\end{description}
\end{block}
\begin{block}{valueN}
\begin{itemize}
	\item One of predefined PAM return values: \\
		success, open\_err, symbol\_err, service\_err, system\_err, buf\_err, perm\_denied, auth\_err, cred\_insufficient, authinfo\_unavail, user\_unknown, maxtries, new\_authtok\_reqd, acct\_expired, session\_err, cred\_unavail, cred\_expired, cred\_err, no\_module\_data, conv\_err, authtok\_err, authtok\_recover\_err, authtok\_lock\_busy, authtok\_disable\_aging, try\_again, ignore, abort, authtok\_expired, module\_unknown, bad\_item, conv\_again, incomplete.
	\item {\bf default}: all PAM return values not mentioned explicitly.
\end{itemize}
\end{block}
\end{frame}

%%%%%%%
\begin{frame}{\Large PAM config rule control values: advanced syntax \hfill [\insertframenumber/\inserttotalframenumber]}
\begin{block}{The syntax: [{\it value1}={\it action1} {\it value2}={\it action2} \ldots {\it valueN}={\it actionN}]}
\begin{description}
	\item[valueN] corresponds to the return value returned by the module
	\item[actionN] specifies the action
\end{description}
\end{block}
\begin{block}{actionN}
\begin{itemize}
	\item {\bf ignore}: return value ignored, stack processing continues
	\item {\bf bad}: module fails, stack processing continues
	\item {\bf die}: module fails, stack processing terminates
	\item {\bf ok}: module succeeds, stack processing continues
	\item {\bf done}: module succeeds; stack processing terminates \\
		if no prior {\bf required} module has failed
	\item {\bf reset}: the stack resets, stack processing continues
	\item {\bf N (an unsigned integer)}: jump over the next N modules in the stack
\end{itemize}
\end{block}
\end{frame}

%%%%%%%
\begin{frame}{\Large PAM config rule control values: advanced syntax \hfill [\insertframenumber/\inserttotalframenumber]}
\begin{block}{The syntax: [{\it value1}={\it action1} {\it value2}={\it action2} \ldots {\it valueN}={\it actionN}]}
If a return value is not specifically listed via a {\it valueN} token, and \\
the {\bf default} value is not specified, the implicit default action for it is {\bf bad}.
\end{block}
\begin{block}{Equivalents of traditional 4 control keywords in the advanced syntax}
\begin{description}
	\item[required] [success={\bf ok} new\_authtok\_reqd={\bf ok} ignore=ignore default={\bf bad}]
	\item[requisite] [success={\bf ok} new\_authtok\_reqd={\bf ok} ignore=ignore default={\bf die}]
	\item[sufficient] [success={\bf done} new\_authtok\_reqd={\bf done} default={\bf ignore}]
	\item[optional] [success={\bf ok} new\_authtok\_reqd={\bf ok} default={\bf ignore}]
\end{description}
\end{block}
\begin{block}{Why use this?}
\begin{itemize}
	\item Complex logic that traditional controls cannot express.
	\item Conditional branching.
\end{itemize}
\end{block}
\end{frame}

%%%%%%%
\begin{frame}[fragile]{\Large PAM config rule control values: advanced syntax \hfill [\insertframenumber/\inserttotalframenumber]}
\begin{block}{Example: /etc/pam.d/system-auth (excerpt)}
\begin{alltt}
\ldots
password  requisite   pam_pwquality.so
password  [success=ok default=1 ignore=ignore] {\textbackslash}
                      pam_localuser.so
password  requisite   pam_pwhistory.so use_authtok
password  sufficient  pam_unix.so shadow nullok use_authtok
password  required    pam_deny.so

session   optional    pam_keyinit.so revoke
session   required    pam_limits.so
session   optional    pam_systemd.so
session   [success=1 default=ignore] {\textbackslash}
                      pam_succeed_if.so service in crond quiet use_uid
session   required    pam_unix.so
\ldots
\end{alltt}
\end{block}
\end{frame}

%%%%%%%
\begin{frame}[fragile]{\Large Frozen stack \hfill [\insertframenumber/\inserttotalframenumber]}
\begin{block}{Example: https://github.com/linux-pam/linux-pam/issues/680}
My custom PAM file:
{\begin{alltt}
\ldots
auth [ignore=1 default=ignore] pam\_env.so envfile=/etc/test\_env
auth required pam\_echo.so "111"
auth required pam\_echo.so "222"
\ldots
\end{alltt}}
My /etc/test\_env:
\begin{alltt}
TEST\_VAR=foo
\end{alltt}
Looks like:
\begin{itemize}
	\item the env variable TEST\_VAR is not set at all
	\item pam\_env.so always return PAM\_IGNORE as I didn't see "111" in the logs
\end{itemize}
\end{block}
\end{frame}

%%%%%%%
\begin{frame}{\Large Frozen stack \hfill [\insertframenumber/\inserttotalframenumber]}
\begin{block}{Applications talk to the PAM library}
\begin{tabular}{l|l|l}
	{\bf type} & {\bf API function name} & {\bf description} \\
\hline
	auth & pam\_authenticate & Authenticate this user \\
	auth & pam\_setcred & Manage credentials of this user \\
\hline
	account & pam\_acct\_mgmt & Check account validity and restrictions for this user \\
\hline
	password & pam\_chauthtok & Change the authentication token for this user \\
\hline
	session & pam\_open\_session & Set up a session for this user \\
	session & pam\_close\_session & End the session for this user \\
\end{tabular}
\end{block}
\begin{block}{Frozen stack}
	The PAM library determines and fixes the list and order of modules for a specific management group (like {\bf auth} or {\bf session}) during the first API call to the stack, \\
	and then reusing that exact same (frozen) sequence of modules for subsequent API calls to this stack.
\end{block}
\end{frame}

%%%%%%%
\begin{frame}[fragile]{\Large Frozen stack \hfill [\insertframenumber/\inserttotalframenumber]}
\begin{block}{Example: https://github.com/linux-pam/linux-pam/issues/680}
My custom PAM file:
{\begin{alltt}
\ldots
auth [ignore=1 default=ignore] pam\_env.so envfile=/etc/test\_env
auth required pam\_echo.so "111"
auth required pam\_echo.so "222"
\ldots
\end{alltt}}
\begin{itemize}
	\item When invoked by pam\_authenticate, pam\_env.so does nothing \\
		and always returns PAM\_IGNORE.
	\item After pam\_authenticate the PAM auth stack is already frozen, \\
		so during pam\_setcred the modules are being called in the same order \\
		as they were called during pam\_authenticate.
\end{itemize}
\end{block}
\end{frame}

%%%%%%%
\begin{frame}{\Large Troubleshooting and best practices \hfill [\insertframenumber/\inserttotalframenumber]}
\Large
\begin{block}{Common issues}
\begin{itemize}
	\item Getting locked out
	\item Incorrect module order or control directives
	\item Syntax errors or typos in config files
	\item Missing module arguments
\end{itemize}
\end{block}
\end{frame}

%%%%%%%
\begin{frame}{\Large Troubleshooting and best practices \hfill [\insertframenumber/\inserttotalframenumber]}
\Large
\begin{block}{Best Practices}
\begin{itemize}
	\item Know what you are doing
	\item Backup before making changes
	\item Always test the changes
	\item Make incremental changes
	\item Always have an emergency root shell open \\
		when testing on a non-disposable system
	\item Use distribution tools, study distribution defaults
\end{itemize}
\end{block}
\end{frame}

%%%%%%%
\begin{frame}{\Large Troubleshooting \hfill [\insertframenumber/\inserttotalframenumber]}
\begin{block}{System logs is the primary tool}
\begin{itemize}
	\item journalctl -u <service-name>
\end{itemize}
\end{block}
\begin{block}{Increase PAM verbosity}
\begin{itemize}
	\item PAM is usually quite verbose already.
	\item Many modules become more verbose with {\bf debug} argument.
\end{itemize}
\end{block}
\begin{block}{RTFM}
\begin{itemize}
	\item Manual pages: PAM(8), pam.conf(5), pam\_*(8).
	\item The Linux-PAM System Administrators' Guide.
	\item The Linux-PAM Module Writers' Guide.
	\item The Linux-PAM Application Developers' Guide.
\end{itemize}
\end{block}
\end{frame}

%%%%%%%
\begin{frame}{\Large Debugging \hfill [\insertframenumber/\inserttotalframenumber]}
\Large
\begin{columns}
	\column{9cm}
	\begin{block}{strace}
	\begin{itemize}
		\item -p \$PID
		\item -f/--follow-forks
		\item -b execve
		\item -r/--relative-timestamps
	\end{itemize}
	\end{block}
	\column{3cm}
		\centerline{\pgfuseimage{strace-logo}}
\end{columns}
\end{frame}

%%%%%%%
\begin{frame}[noframenumbering]{Questions?}
\end{frame}

\end{document}
